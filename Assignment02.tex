\documentclass{report}

\title{Theory and Technology of Semiconductor Devices\\ Assignment 2 \\ \vspace{30pt} Dr. Shooshtari}

\author{}
\date{Due: Dec 22}


\begin{document}
	\maketitle
	
	\begin{enumerate}
		\item[\bf{Problem 1}] Doing an Arsenic diffusion on a p-type wafer with background doping of $10^{17} cm^{-3}$ with limited amount of Arsenic (constant-total-dopant) of $10^{14} cm^{-2}$ at $1100^{\circ} C$ for 1.5 hours, find the surface concentration and junction depth. \\
		Note: $D=2.07\times 10^{-14}$ $cm^2/s$ for Arsenic at $1100^{\circ} C$
		
		\item[\bf{Problem 2}] A two-step Boron diffusion is used on an n-type wafer to form a p-type region; first 30 minutes with constant surface concentration of $1.1\times 10^{20} cm^{-3}$ at $900^{\circ} C$ followed by 4 hours (the constant-total step) at $1100^{\circ} C$. find (a) junction depth (b) the gradient of concentration for $x=x_j$ at the end of each step (c) the surface concentration at the end of step 2\\
		Note: $900^{\circ} C \rightarrow D = 1.45\times 10^{-15}$ $cm^{2}/s$ \\
		$1100^{\circ} C \rightarrow D = 2.96\times 10^{-13}$ $cm^{2}/s$ \\
		Note: solving step 2 has Bonus mark (not required)
	\end{enumerate}
\end{document}